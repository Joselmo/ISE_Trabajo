\input{preambuloSimple.tex}

%----------------------------------------------------------------------------------------
%	TÍTULO Y DATOS DEL ALUMNO
%----------------------------------------------------------------------------------------

\title{	
\normalfont \normalsize 
\textsc{\textbf{Ingeniería de Servidores (2016-2017)} \\ Grado en Ingeniería Informática \\ Universidad de Granada} \\ [25pt] % Your university, school and/or department name(s)
\horrule{0.5pt} \\[0.4cm] % Thin top horizontal rule
\huge systemd \\ % The assignment title
\horrule{2pt} \\[0.5cm] % Thick bottom horizontal rule
}

\date{\normalsize\today} % Incluye la fecha actual

%----------------------------------------------------------------------------------------
% DOCUMENTO
%----------------------------------------------------------------------------------------

\begin{document}

\maketitle % Muestra el Título

\newpage

%----------------------------------------------------------------------------------------
%	Resumen introductorio
%----------------------------------------------------------------------------------------
\begin{abstract}

En el mundo de \textit{UNIX} donde el kernel está público y 
cada vez es más grande la comunidad que desarrolla nuevas tecnologías 
para mejorar el sistema \textit{UNIX}, en nuestro caso vamos a analizar
una tecnología que rápidamente ha sido adaptada por las principales 
distribuciones de \textit{UNIX}, systemd. Se explicará en que consiste esta 
tecnología y sus ventajas e inconvenientes. Además examinaremos la
 implementación de systemd para saber un poco más sobre como funciona.

\end{abstract}

%----------------------------------------------------------------------------------------
%	Introducción
%----------------------------------------------------------------------------------------

\section{Introducción} % máximo 2 páginas
En los sistemas de tipo \textit{UNIX}, concedidos al principio de la decada
de los 70, como se indica en la historia de The Open Group \cite{unix}, 
se ideo de forma gratuita y libre. Fomentando la idea de software libre
y de código abierto y creando la comunidad tan grande y amplia que conocemos 
que conocemos hoy en día. Gracias a esta comunidad se ha seguido
desarrollando y avanzando el sistema \textit{UNIX} hasta el kernel de 
\textit{Linux} que conocemos y utilizamos a diario. Constantemente se están
desarrollando nuevas tecnologías y procesos para mejorar el sistema.

Nosotros nos vamos a centrar en comparar una nueva versión de unos de los procesos principales del sistema con su predecesora que curiosamente se ha
 mantenido casi intacto 
desde la década de los noventa, es decir, desde hace más de 20 años no
ha sido modificado, echo sorprendente pensando en la velocidad que avanza
siempre la tecnología. El proceso en cuestión es el encargado del inicio 
del sistema, el proceso con PID (Proccess ID) 1, el arranque del sistema 
lo hace ejecutando todos los demás procesos como los drivers del adaptador
de red o del ratón, los controladores de pantalla, etc. 

Hasta hace relativamente poco todos los sistemas basados en Linux 
utilizaban en su kernel el proceso \textit{init.d}. Init.d se caracterizaba
por su simpleza y facilidad de uso. El funcionamiento de init.d consiste
en ir iniciando los procesos listados en un archivo de configuración, 
es decir, inicia el primer proceso de la lista y cuando este se ha iniciado
inicia el siguiente y así sucesivamente. De esta forma tan simple se
iniciaba el sistema. Además utiliza la filosofía del software libre de 
ser transparente para el usuario y permitirle poder modificar por completo
su implementación, ya que bastaba con modificar el fichero de configuración.

Pero tenia unos claros inconvenientes, primero las dependencias
deben controlarse por parte del programador que tiene que saber que procesos
van han de iniciarse primero y cuales después, por ejemplo si en el listado
de procesos aparece un proceso apache que depende de la configuración de
red este proceso se iniciará con errores o no se iniciará puesto que el 
controlador de red todavía no se ha iniciado, aun cuando apache este 
perfectamente configurado y el controlador de red también. El otro
problema principal era la sobrecarga al inicio del sistema, ya que se 
iniciaban los procesos de uno en uno y hasta que uno no acabase de arrancar
no se iniciaba el siguiente.

Por ello Lennart Poettering, como redacta en su articulo \cite{Lennart} y
Kay Sievers decidieron desarrollar uno proceso nuevo de arranque del
sistema, desarrollando así \textit{systemd}.



%----------------------------------------------------------------------------------------
%	Systemd
%----------------------------------------------------------------------------------------

\section{Systemd} % máximo 10 páginas
\subsection{¿Qué es systemd?}
Tal como la definen los autores en la web oficial de systemd \cite{systemd}
es una suite o conjunto de herramientas diseñadas para ofrecer una 
funcionalidad específica, en este caso para facilitar y mejorar el arranque
del sistema operativo \textit{UNIX}. Específicamente es un conjunto de demonios de \textit{UNIX}, un demonio \cite{daemons} es un proceso que
funciona en segundo plano en el sistema a la espera de eventos o llamadas
que se producen en el sistema y cuando son despertados realizan una tarea 
concreta. 



\subsection{Historia}

\subsubsection{Sobre su creador}

\subsection{Funcionamiento}

\subsection{Implementación}

\subsection{Conclusiones}
%----------------------------------------------------------------------------------------
%	Conclusiones
%----------------------------------------------------------------------------------------

\section{Conclusiones}

\begin{figure}[H] %con el [H] le obligamos a situar aquí la figura
\centering
\includegraphics[scale=0.4]{./imagenes/P3_4_1.png} 
\caption{Informe del monitor de rendimiento - resumen} \label{fig:P3_4_1}
\end{figure}




%------------------------------------------------

\bibliography{citas} %archivo citas.bib que contiene las entradas 
\bibliographystyle{plain} % hay varias formas de citar

\end{document}



