\input{preambuloSimple.tex}

%----------------------------------------------------------------------------------------
%	TÍTULO Y DATOS DEL ALUMNO
%----------------------------------------------------------------------------------------

\title{	
\normalfont \normalsize 
\textsc{\textbf{Ingeniería de Servidores (2016-2017)} \\ Grado en Ingeniería Informática \\ Universidad de Granada} \\ [25pt] % Your university, school and/or department name(s)
\horrule{0.5pt} \\[0.4cm] % Thin top horizontal rule
\huge Memoria \\ % The assignment title
systemd
\horrule{2pt} \\[0.5cm] % Thick bottom horizontal rule
}

\date{\normalsize\today} % Incluye la fecha actual

%----------------------------------------------------------------------------------------
% DOCUMENTO
%----------------------------------------------------------------------------------------

\begin{document}

\maketitle % Muestra el Título

\newpage

%----------------------------------------------------------------------------------------
%	Resumen introductorio
%----------------------------------------------------------------------------------------

\section{Resumen}% Entre 5 y 10 lineas
En el mundo de \textit{UNIX} donde cada parte del kernel está público y 
cada vez es más grande la comunidad que desarrolla nuevas tecnologías 
para mejorar el sistema \textit{UNIX}, en nuestro caso vamos a analizar
una tecnología que rápidamente ha sido adaptada por las principales 
distribuciones de \textit{UNIX}, systemd. Se explicará en que consiste esta 
tecnología y sus ventajas e inconvenientes. Además examinaremos la
 implementación de systemd para saber un poco más sobre como funciona.

%----------------------------------------------------------------------------------------
%	Introducción
%----------------------------------------------------------------------------------------

\section{Introducción} % máximo 2 páginas


%----------------------------------------------------------------------------------------
%	Systemd
%----------------------------------------------------------------------------------------

\section{Systemd} % máximo 10 páginas
\subsection{¿Qué es systemd?}
Tal como la definen los autores en la web oficial de systemd \cite{systemd}
es una suite o conjunto de herramientas diseñadas para ofrecer una 
funcionalidad específica, en este caso para facilitar y mejorar el arranque
del sistema operativo \textit{UNIX}
%----------------------------------------------------------------------------------------
%	Conclusiones
%----------------------------------------------------------------------------------------

\section{Conclusiones}

\begin{figure}[H] %con el [H] le obligamos a situar aquí la figura
\centering
\includegraphics[scale=0.4]{./imagenes/P3_4_1.png} 
\caption{Informe del monitor de rendimiento - resumen} \label{fig:P3_4_1}
\end{figure}




%------------------------------------------------

\bibliography{citas} %archivo citas.bib que contiene las entradas 
\bibliographystyle{plain} % hay varias formas de citar

\end{document}



