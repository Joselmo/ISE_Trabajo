\input{preambuloSimple.tex}

%----------------------------------------------------------------------------------------
%	TÍTULO Y DATOS DEL ALUMNO
%----------------------------------------------------------------------------------------

\title{	
\normalfont \normalsize 
\textsc{\textbf{Ingeniería de Servidores (2016-2017)} \\ Grado en Ingeniería Informática \\ Universidad de Granada} \\ [25pt] % Your university, school and/or department name(s)
\horrule{0.5pt} \\[0.4cm] % Thin top horizontal rule
\huge Memoria \\ % The assignment title
\horrule{2pt} \\[0.5cm] % Thick bottom horizontal rule
}

\author{Elena María Gómez Ríos y Jose Luis Martínez Ortiz} % Nombre y apellidos

\date{\normalsize\today} % Incluye la fecha actual

%----------------------------------------------------------------------------------------
% DOCUMENTO
%----------------------------------------------------------------------------------------

\begin{document}

\maketitle % Muestra el Título

\newpage %inserta un salto de página

\tableofcontents % para generar el índice de contenidos

\listoffigures

\listoftables

\newpage

%\textbf{NOTA: en caso de problema al compilar, compruebe que tiene el paquete: texlive-babel-spanish.noarch }  \\
 
\newpage

%----------------------------------------------------------------------------------------
%	Resumen introductorio
%----------------------------------------------------------------------------------------

\section{Resumen}


%----------------------------------------------------------------------------------------
%	Introducción
%----------------------------------------------------------------------------------------

\section{Introducción}


%----------------------------------------------------------------------------------------
%	Systemd
%----------------------------------------------------------------------------------------

\section{Systemd}


%----------------------------------------------------------------------------------------
%	Conclusiones
%----------------------------------------------------------------------------------------

\section{Conclusiones}

\begin{figure}[H] %con el [H] le obligamos a situar aquí la figura
\centering
\includegraphics[scale=0.4]{./imagenes/P3_4_1.png} 
\caption{Informe del monitor de rendimiento - resumen} \label{fig:P3_4_1}
\end{figure}


%----------------------------------------------------------------------------------------
%	Tareas extra
%----------------------------------------------------------------------------------------
\section{Tareas extra.}


\subsection{Instalación y utilización de lm-sensors}
Para probar el software que se indica en el guión de prácticas \textit{lm-sensor} consulto 
la wiki de Arch \cite{lmSensor}, ya que la web indicada en el guión no esta operativa.
Lo realizo sobre mi propio equipo ya que en las máquinas virtuales no cuentan con los sensores de
las BIOS.
Como se puede ver en la imagen \ref{fig:P3_E2_1} ya tengo instalado el software. Ahora paso a 
configurar a que sensores de mi equipo le doy acceso para monitorizar (figura \ref{fig:P3_E2_2}).
Una vez configurado los permisos ya podemos ver la información mediante el comando \textit{sensors},
tal como se ve en la figura \ref{fig:P3_E2_3}.\\

Podemos ver como la temperatura de los ``cores'' esta en valores normales, además nos indica que por
encima de 81º es una temperatura alta y por encima de los 105º ya es critica. También nos informa de la
velocidad del ventilador del equipo, en nuestro caso esta funcionando a 27000 rpm (revoluciones por minuto)




%------------------------------------------------

\bibliography{citas} %archivo citas.bib que contiene las entradas 
\bibliographystyle{plain} % hay varias formas de citar

\end{document}



