\documentclass[11pt]{beamer}
\usetheme{Copenhagen}
\usepackage[utf8]{inputenc}
\usepackage[spanish]{babel}
\usepackage{amsmath}
\usepackage{amsfonts}
\usepackage{amssymb}
\usepackage{graphicx}
\author{Elena María Gómez Ríos y Jose Luis Martínez Ortiz}
\title{Comparativa entre Systemd e init}
%\setbeamercovered{transparent} 
%\setbeamertemplate{navigation symbols}{} 
%\logo{} 
%\institute{} 
%\date{} 
%\subject{} 
\begin{document}

\begin{frame}
\titlepage
\end{frame}

\begin{frame}
\begin{enumerate}
\item ¿Qué es Init?
\item System V init
\item ¿Qué es Systemd?
\item Estructura
\item Demonios
\item Ventajas
\item Desventajas
\item Controversia generada
\item Conclusiones
\end{enumerate}
\end{frame}


\begin{frame}{¿Qué es Init?} % 1
Es el proceso encargado del inicio de sistemas \texttt{Linux}.\\
\vspace{1cm}
El proceso init se caracteriza por su simpleza y facilidad de uso. El funcionamiento de init consiste en ir iniciando los procesos listados en un archivo de configuración, es decir, inicia el primer proceso del listado y cuando éste se ha iniciado inicia el siguiente y así sucesivamente.
\end{frame}
% -----------------------------------------------------
\begin{frame}{System V init} % 2
System V es una versión mejorada del init original, que consiste en un sistema de llamadas por niveles de prioridad para la ejecución de los procesos del sistema.
\end{frame}
% -----------------------------------------------------
\begin{frame}{¿Qué es Systemd?} % 3
d
\end{frame}
% -----------------------------------------------------
\begin{frame}{Estructura} % 4
d
\end{frame}
% -----------------------------------------------------
\begin{frame}{Demonios} % 5
d
\end{frame}
% -----------------------------------------------------
\begin{frame}{Ventajas} % 6
d
\end{frame}
% -----------------------------------------------------

\begin{frame}{Desventajas} % 7
d
\end{frame}

% -----------------------------------------------------
\begin{frame}{Controversia generada} % 8
d
\end{frame}
% -----------------------------------------------------
\begin{frame}{Conclusiones} % 9
d
\end{frame}


\end{document}